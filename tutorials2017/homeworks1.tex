\documentclass[12pt]{article}
 
\usepackage[margin=1in]{geometry} 
\usepackage{amsmath,amsthm,amssymb}
\usepackage{hyperref}
 
\begin{document}
\title{Introduction to Particle Accelerators\\
\large Homeworks I}
\author{}
\date{}
 
\maketitle
 
\section{Magnetic rigidity}
\label{problem1}
Derive the expression of the magnetic rigidity $B\rho$ as a function of the particle relativistic momentum and charge:
\begin{equation}
B \rho = \frac{p}{q}.
\end{equation}

Express it numerically for protons and for electrons with $B$ in Tesla, $\rho$ in meter and $p$ in $\text{GeV/c}$.

\section{Relativistic kinematics}
\label{problem2}
Compute the following quantities for protons at injection energies of the CERN PB Booster (PBS) (injection kinetic energy of $50$ MeV), PS (injection kinetic energy of $1.4$ GeV), SPS (injection kinetic energy of $26$ GeV) and LHC (injection kinetic energy of $450$ GeV):
\begin{itemize}
\item Momentum;
\item Kinetic energy, total energy;
\item Relativistic $\beta$ and $\gamma$;
\item Magnetic rigidity (see problem \ref{problem1}).
\end{itemize}

\section{Ion beams for the LHC}
Assuming the same magnetic fields as for problem \ref{problem2}, compute the injection energies for Xenon and Lead ion beams\footnote{See \url{https://home.cern/cern-people/updates/2017/10/lhc-report-xenon-action}.} (total, per nucleon and per charge).

\section{Betatron 2-in-1 rule}
Derive the 2-in-1 rule for constant orbit in a betatron (Wideroe's condition). Why is it important to consider the radial dependence of the magnetic field.
 
\end{document}